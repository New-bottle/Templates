%文档类型
\documentclass[a4paper]{article}

%引用包裹
\usepackage{bm}
\usepackage{cmap}
\usepackage{ctex}
\usepackage{cite}
\usepackage{color}
\usepackage{float}
\usepackage{xeCJK}
\usepackage{amsthm}
\usepackage{amsmath}
\usepackage{amssymb}
\usepackage{setspace}
\usepackage{geometry}
\usepackage{hyperref}
\usepackage{enumerate}
\usepackage{indentfirst}
\usepackage[cache=false]{minted}

%代码高亮
\geometry{margin=1in}

%字体设置
%\setmainfont{PingFangSC-Light}
%\setCJKmonofont{PingFangSC-Light}
%\setCJKmainfont[BoldFont={PingFangSC-Regular}]{PingFangSC-Light}


\newcommand{\cppcode}[1]{
    \inputminted[mathescape,
    			tabsize=2
    			]{cpp}{source/#1}
}

\newcommand{\javacode}[1]{
    \inputminted[mathescape,
    			tabsize=2
    			]{java}{source/#1}
}

\title{代码库}
\author{上海交通大学}
\date{\today}

\begin{document}

\maketitle

\tableofcontents

\clearpage

\section{数论}
	\subsection{中国剩余定理}
		\cppcode{maths/CRT.cpp}

	\subsection{FFT}
		\cppcode{maths/FFT.cpp}

	\subsection{组合数学(卢卡斯定理、线性筛逆元)}
		\cppcode{maths/Combination.cpp}

	\subsection{辛普森自适应积分}
		\cppcode{maths/simpson.cpp}

	\subsection{线性筛}
		\cppcode{maths/linearsieve.cpp}
\section{图论}
	\subsection{Tarjan}
		\cppcode{graph/tarjan.cpp}

	\subsection{最大流}
		\cppcode{graph/dinic.cpp}

	\subsection{最小费用流}
		\cppcode{graph/mincostflow.cpp}
\section{数据结构}
	\subsection{K-D树}
		\cppcode{data_structure/KDTree.cpp}

	\subsection{可持久化Trie}
		\cppcode{data_structure/persistant-trie.cpp}

	\subsection{Link-Cut-Tree}
		\cppcode{data_structure/LCT.cpp}

	\subsection{可持久化线段树}
		\cppcode{data_structure/persistant-segtree.cpp}

\section{字符串}
	\subsection{AC自动机}
		\cppcode{strings/ACautomaton.cpp}

	\subsection{后缀数组}
		\cppcode{strings/suffixarray.cpp}

	\subsection{回文自动机}
		\cppcode{strings/PAM.cpp}

	\subsection{Manacher算法}
		\cppcode{strings/manacher.cpp}
\end{document}
